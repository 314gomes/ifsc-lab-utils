\documentclass[a4paper, 12pt]{article}
\usepackage[utf8]{inputenc}
\usepackage[portuguese]{babel}
\usepackage{indentfirst}
\usepackage{amsmath}
\usepackage{multicol}


\title{Relatório\\
Laboratório de Físsica II\\
Prática 1: Rotações de corpos rígidos e conservação do momento angular}

\author{Felipe Barbetti de Grabalos\\
\texttt{nro usp felipe}
\and
João Pedro Gomes\\
\texttt{13839069}
\and
Maria Victoria Barreto Teixeira\\
\texttt{nro usp mavi}
}
\date{Setembro 2022}

\begin{document}
\maketitle

\section{Objetivos}
    Esta prática visa estudar conceitos de rotação de corpos rígidos, a fim de comprovar empiricamente conceitos já vistos em sala de aula. Destacam-se a comprovação do momento de inércia e a análise da conservação do momento angular na colisão rotacional plástica.

\section{Hipóteses}
    Definimos que um corpo é rígido quando é indeformável, isto é, a distância entre quaisquer duas partículas do corpo permanece constante. Objetos reais não são perfeitamente rígidos, mas deformações em certas condições são negligenciáveis.
    \par
    Definimos que um corpo está rodando em torno de um eixo fixo quando a alteração angular de cada partícula é descrita em função do tempo. Define-se então $r$ como o raio da partícula em relação ao eixo, $s$ como o espaço percorrido na trajetória, $\theta$ como a posição angular (em radianos) de uma partícula em relação a uma origem, $\omega$ como a velocidade angular e $\alpha$ como aceleração angular, onde
    \begin{equation}
        \label{posang}
        \theta = \frac{s}{r}
    \end{equation}
    \begin{equation}
        \label{velang}
        \omega = \frac{d}{dt}\theta
    \end{equation}
    \begin{equation}
        \label{acang}
        \alpha = \frac{d}{dt}\omega
    \end{equation}
    \par
    Entretanto, essas equações não são suficientes para modelar diversas situações. Para descrever a energia de um corpo em rotação, por exemplo, é preciso considerar a velocidade tangencial de cada uma de suas partículas. A energia cinética $K$ pode então ser descrita com o somatório $K = \sum \frac{1}{2} m_iv_i^2$. Porém, substituindo \eqref{posang} em \eqref{velang} e isolando $v$ pode-se obter o novo somatório $K = \sum \frac{1}{2} m_i(\omega r_i) = \frac{1}{2} (\sum m_i r_i^2) \omega^2$.
    \par
    Define-se então o momento de inércia de um corpo como o somatório $\sum m_i r_i^2$, que pode ser substituído pela integral \eqref{mominerc} para corpos contínuos.
    \begin{equation}
        \label{mominerc}
        I=\int r^2\, dm
    \end{equation}
    \par
    Desta forma, a expressão para energia cinética rotacional passa a ser \eqref{krot}.
    \begin{equation}
        \label{krot}
        K_\textbf{rot} = \frac{1}{2} I \omega
    \end{equation}
    \par
    Isto forma nossas duas primeiras hipóteses:
    \begin{description}
        \item[Momento de inércia]{Faz sentido descrever o momento de inércia de um corpo com a equação \eqref{mominerc} e essa grandeza pode ser calculada de diferentes formas.}
        \item[Energia cinética rotacional]{A energia cinética rotacional de um corpo é descrita por \eqref{krot}}.
    \end{description}
    \par
    Outra grandeza de grande auxílio no estudo de rotações é o momento angular $\ell$ descrito por $\ell = r \times p = m(r \times v)$ onde $p$ é o momentum de uma partícula e $v$ sua velocidade tangencial.
    \par
    No caso de corpos contínuos, utiliza-se uma dedução similar a \eqref{krot} para definir seu momento angular \eqref{momang}.
    \begin{equation}
        \label{momang}
        L = I\omega
    \end{equation}
    \par
    Por si mesmo, o momento angular não é muito útil já que é uma grandeza axiomática. Entretanto, pode ser demonstrado que essa definição implica em \eqref{2leiang}, onde $\sum\tau$ é o torque total no corpo. 
    \begin{equation}
        \label{2leiang}
        \sum\tau = \frac{dL}{dt}
    \end{equation}
    \par
    Disso, surge nossa terceira hipótese:
    \begin{description}
        \item[Conservação do momento angular]{Num sistema com torque total nulo, o momento angular se conserva.}
    \end{description}

\section{Materiais}
    \begin{enumerate}
        \item{}
    \end{enumerate}

\section{Métodos}
    medir as coisas com régua e aferir as hipóteses

\section{Fontes de erros}

\section{Resultados}
    resultados comprovam que $b=a^x$

\section{Discussão}
    matemática faz sentido

%---------------------%
%REFERÊNCIAS
%---------------------%
\pagebreak
\bibliographystyle{abntex2-num.bst}
\bibliography{bibliografia}

\nocite{apostilai}
\nocite{apostilaii}
\nocite{nussenzveig2013curso}
\nocite{halliday2013fundamentals}

\end{document}
